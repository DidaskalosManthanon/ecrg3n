\subsection{Exercice 10 (secondaire : collège) : Un programme de construction géométrique, Pythagore}

\label{geom:niveau10}

\begin{enumerate}[label=G\arabic*)]
	\item Dans un repère orthonormé placer le point \(A\) de coordonnées \((-1 ; -2)\) c'est-à-dire d'abscisse \(x_A = -1\) et d'ordonnée \(y_A = -2\)
	\item Placer le point \(B\) de coordonnées \((2 ; -2)\) c'est-à-dire d'abscisse \(x_B = 2\) et d'ordonnée \(y_B = -2\)
	\item Placer le point \(C\) de coordonnées \((2 ; 2)\) c'est-à-dire d'abscisse \(x_C = 2\) et d'ordonnée \(y_C = 2\)
	\item En utilisant le théorème de Pythagore vérifier que le carré ABC est rectangle en B.
\end{enumerate}