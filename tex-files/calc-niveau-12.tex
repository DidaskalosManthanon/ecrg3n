\subsection{Exercice 12 (secondaire : collège) : carrés de Fibonacci}

\label{geom:niveau12}

\begin{enumerate}[label=G\arabic*)]
	\item Dans un repère orthonormé construire le carré passant par les points O(0; 0) , A(1 ; 0) , B(1 ; 1) , C(0 ; 1). Quelle est la longueur du côté de carré ?
	\item Placer les points D(2 ; 0) et E(2 ; 1) et tracer le carré ADEB. Quelle est la longueur du côté de carré ?
	\item Construire le carré passant par F(2 ; 3) , G(0 ; 3) , C(0 ; 1), E(2 ; 1). Quelle est la longueur du côté de carré ?
	\item Placer les points H$(-3 ; 3)$, I$(-3 ; 0)$ et tracer le carré GHIO. Quelle est la longueur du côté de carré ?
	\item Construire le carré passant par J$(-3 ; -5)$, K$(2 ; -5)$, D$(2 ; 0)$, I$(-3 ; 0)$. Quelle est la longueur du côté de carré ?
	\item Placer les points L$(10 ; -5)$, M$(10 ; 3)$ et tracer le carré KLMF. Quelle est la longueur du côté de carré ?
	\item Construire le carré passant par N$(10 ; 16)$, P$(-3 ; 16)$ et tracer le carré MNPH. Quelle est la longueur du côté de carré ?
	\item Placer les points Q$(-24 ; 16)$, R$(-24 ; -5)$ et tracer le carré PQRJ. Quelle est la longueur du côté de carré ?
\end{enumerate}