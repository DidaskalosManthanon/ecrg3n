\section{Que faire une fois que vous avez fait tous les exercices ?}
\addcontentsline{toc}{chapter}{Que faire une fois que vous avez fait tous les exercices ?}

\subsection{Vous en voulez encore ?}

Voilà, c'est terminé vous avez achevé tous les exercices de cet ouvrage. Bravo ! Maintenant, que faire ? 

Vous avez plusieurs possibilités. Si vous êtes créatif et curieux vous pouvez commencer par modifier les exercices proposés ici. C'est un très bon moyen d'approfondir. Ensuite vous pouvez aller voir du côté de mes exercices en ligne : 

\begin{itemize}
	\item \urlnote{Comprendre la notion de fonction affine}{https://didaskalosmanthanon.github.io/fct-affine/}
	\item \urlnote{Calcul de l'image par une fonction affine}{https://didaskalosmanthanon.github.io/image-fonction-affine/index.html}
	\item \urlnote{Antécédent par une fonction affine}{https://didaskalosmanthanon.github.io/antecedent-fonction-affine/index.html}
	\item \urlnote{QCM sur divers thèmes}{https://didaskalosmanthanon.github.io/quizz/index.html}  : trigonométrie, identités remarquables...
	\item \urlnote{QCM sur les statistiques en 3ème}{https://didaskalosmanthanon.github.io/quizz-stats-3eme/index.html}
	\item \urlnote{QCM sur les nombres en 3ème}{https://didaskalosmanthanon.github.io/qcm-numbers/}
	\item \urlnote{QCM sur les fonctions en classe de 3ème}{https://didaskalosmanthanon.github.io/quizz-fonctions-3eme/index.html}
	\item \urlnote{Exercices de géométrie niveau 3ème - 2de}{https://didaskalosmanthanon.github.io/geometry/exercice1.html}
	\item \urlnote{Ce que mes élèves pensent de mes cours}{https://didaskalosmanthanon.github.io/testimonial-slider/index.html}
\end{itemize}

\subsection{D'autres livres du même auteur}

Vous pouvez également vous procurez mes autres ouvrages disponibles sur Amazon : 

\begin{itemize}
	\item \urlnote{Apprendre Python en moins de 5 heures}{https://amzn.to/40n8rdI} un livre pour vous mettre toute suite en action (évidemment que vous ne maîtriserez pas le langage en seulement 5 heures)
	\item \urlnote{Comment devenir autonome en anglais en 3 mois}{https://amzn.to/3THL1Mq} aujourd'hui l'anglais est un outil indispensable que ça soit pour travailler, pour les sciences, pour voyager, pour de la documentation technique et même pour explorer les mathématiques de façon internationale
\end{itemize}


\subsection{Pour allez plus loin}


Pour toutes remarques, suggestions, ou demandes d'aide personnelles merci de remplir \urlnote{ce formulaire}{https://forms.gle/x7fAce7GqiJAGbsC7} : \url{https://forms.gle/x7fAce7GqiJAGbsC7}

En remplissant ce formulaire vous me permettrez de vous contacter pour vous faire part de mes actualités et de l'aide que je pourrais vous apportez si vous en avez besoin. Et vous pourrez également obtenir la version numérique de ce livre.


À très bientôt pour de nouvelles aventures et encore merci d'avoir acheté ce livre et de l'avoir lu jusqu'au bout.

\newpage
