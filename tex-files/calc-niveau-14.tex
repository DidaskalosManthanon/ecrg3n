\subsection{Exercice 14 (secondaire : collège) : une cible avec des carrés de Fibonacci}

\label{proba:niveau14}

Dans cet exercice on continue avec la figure des carrés de Fibonacci voir page \pageref{sol:niveau12}.
On la considère comme une cible particulière et on admet que le joueur atteint forcément le grand rectangle.

\begin{enumerate}[label=G\arabic*)]
	\item Quelle est la probabilité que le joueur atteigne le carré \textcolor{red}{OABC} ? 
	\item Quelle est la probabilité que le joueur atteigne le carré \textcolor{orange}{FGCE} ?
	\item Quelle est la probabilité que le joueur atteigne le carré \textcolor{olive}{GHIO} ?
	\item Quelle est la probabilité que le joueur atteigne le carré \textcolor{navy}{IJKD} ?
	\item Quelle est la probabilité que le joueur atteigne le carré \textcolor{pink}{KLMF} ?
	\item Quelle est la probabilité que le joueur atteigne le carré \textcolor{lime}{MNPH} ?
	\item Quelle est la probabilité que le joueur atteigne le carré \textcolor{purple}{PQRJ} ?    
\end{enumerate}