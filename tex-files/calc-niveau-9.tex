\subsection{Exercice 9 (secondaire : collège) : Un programme de calcul particulier}

\label{calc:niveau9}

Considérons le programme de calcul suivant :

\begin{enumerate}[label=P\arabic*)]
	\item Choisir un nombre entier strictement supérieur à 1 (\(m > 1\) par exemple 2, 3, 4\dots ) 
	\item Choisir un autre nombre entier strictement positif \(n\) strictement inférieur à \(m\quad n < m\)
	\item Calculer le nombre \(a\) comme la différence du carré de \(m\) avec le carré de \(n\), concrètement \(a = m^2 - n^2\)
	\item Calculer le nombre \(b\) comme le double du produit de \(m\) et de \(n\), concrètement \(b = 2mn\)
	\item Calculer le nombre \(c\) comme la somme du carré de \(m\) avec le carré de \(n\), concrètement \(c = m^2 + n^2\)
	\item Calculer le carré de \(a\)
	\item Calculer le carré de \(b\)
	\item Calculer le carré de \(c\)
	\item Calculer la somme du carré de \(a\) et celui de \(b\)
	\item Comparer cette somme avec le carré de \(c\)
\end{enumerate}

\newpage

Appliquons le programme de calcul ci-dessus avec les nombres \(m = 2\) et \(n = 1\) : 

\begin{enumerate}[label=C\arabic*)]
    \item \(m^2 = 2 \times 2 =  \_\_\_\)
    \item \(n^2 = 1 \times 1 = \_\_\_\)
    \item \(a = m^2 - n^2 = \_\_\_\)
    \item \(b = 2 \times m \times n = \_\_\_\)
    \item \(c = m^2 + n^2 = \_\_\_\)
    \item \(a^2 =  \_\_\_\)
    \item \(b^2 =  \_\_\_\)
    \item \(c^2 =  \_\_\_\)
    \item Vérifiez que \( a^2 + b^2 = c^2 \)
    \item Est-ce valable pour n'importe quelles valeurs de \(m\) et \(n\) ?
\end{enumerate}
