\section{Calculs niveau lycée}
\addcontentsline{toc}{chapter}{Calculs niveau lycée}

Désormais il faudrait raisonner et alterner les registres. Tantôt vous utiliserez le langage symbolique avec les formules et le calcul littéral, tantôt vous utiliserez le langage verbal qui est votre langage naturel et tantôt vous utiliserez le langage visuel. Faire des mathématiques consiste principalement à passer d'un registre à un autre afin de s'assurer que l'on comprenne et maîtrise tous les aspects d'un problème. Parfois les choses sembleront abstraites mais il y aura toujours des applications concrètes. 

Courage, la bravoure est une qualité nécessaire pour faire des mathématiques.

\newpage

\subsection{Exercice 15 (secondaire : lycée) : nombres triangulaires}

\label{geom:niveau15}

On considère un repère orthonormé :

\begin{enumerate}[label=G\arabic*)]
	\item Construire le triangle passant par les points O(0; 0) , A(1 ; 0) , B(0 ; 1). Quelle est la nature de ce triangle ? Quelle est l'aire de ce triangle ?
	\item Construire le triangle passant par les points O(0; 0) , C(2 ; 0) , E(0 ; 2). Quelle est la nature de ce triangle ? Quelle est l'aire de ce triangle ? Le point D(1 ; 1) est-il sur le segment [CE] ?
	\item Construire le triangle passant par les points O(0; 0) , F(3 ; 0) , I(0 ; 3). Quelle est la nature de ce triangle ? Quelle est l'aire de ce triangle ? Les points G(2 ; 1) et H(1 ; 2) sont-ils sur le segment [FI] ?
	\item Construire le triangle passant par les points O(0; 0) , J(4 ; 0) , N(0 ; 4). Quelle est la nature de ce triangle ? Quelle est l'aire de ce triangle ? Les points K(3 ; 1), L(2 ; 2) et M(1 ; 3) sont-ils sur le segment [JN] ?
	\item Combien faudra-t-il ajouter de points pour la prochaine étape si on suit ce même schéma ? Quelle sera la nature de ce nouveau triangle ? Quelle sera son aire ? Les points entre les axes du repère seront-ils alignés ?
\end{enumerate}

\hyperref[sol:niveau15]{Voir solutions page \pageref{sol:niveau15}}.

\newpage

\subsection{Exercice 16 (secondaire : lycée) : divisions par 7}

\label{calc:niveau16}

\begin{enumerate}[label=C\arabic*)]
	\item Effectuer la division décimale de 1 par 7. Combien de décimales avez-vous besoin de calculer pour approcher la fraction $\dfrac{1}{7}$ de la façon la plus juste qui soit ?
	\item Même question avec la fraction $\dfrac{2}{7}$.
	\item Même question avec la fraction $\dfrac{3}{7}$.
	\item Même question avec la fraction $\dfrac{4}{7}$.
	\item Même question avec la fraction $\dfrac{5}{7}$.
	\item Même question avec la fraction $\dfrac{6}{7}$.
\end{enumerate}

\hyperref[sol:niveau16]{Voir solutions page \pageref{sol:niveau16}}.


\newpage

\subsection{Exercice 17 (secondaire : lycée) : comment choisir le milieu d'une série statistique ?}

\label{calc:niveau17}

On considère la série statistique $S_0 = \{1 ; 1 ; 1 ; 1 ; 10\}$

\begin{enumerate}[label=C\arabic*)]
	\item Quelle est la médiane de  la série $S_0$ ?
	\item Quel est le mode de  la série $S_0$ ? 
	\item Quelle est la moyenne de  la série $S_0$ ?
	\item Reprendre les 3 questions initiales avec la nouvelle série $S_1 = \{1 ; 1 ; 1 ; 1 ; 10 ; 10\}$. 
	\item Reprendre les 3 questions initiales avec la nouvelle série $S_2 = \{1 ; 1 ; 1 ; 1 ; 10 ; 10 ; 11\}$. 
	\item Reprendre les 3 questions initiales avec la nouvelle série $S_3 = \{1 ; 1 ; 1 ; 1 ; 10 ; 10 ; 11 ; 15 ; 15 ; 15\}$. 
	\item Reprendre les 3 questions initiales avec la nouvelle série $S_4 = \{1 ; 1 ; 1 ; 1 ; 10 ; 10 ; 11 ; 15 ; 15 ; 15 ; 15 ; 15 ; 16 ; 16 ; 18 ; 18\}$. 
	\item Peut-on ajouter 1 valeur à la série $S_4$ de sorte que la médiane augmente et la moyenne diminue ? Expliquez la démarche.
	\item Peut-on ajouter 1 valeur à la série $S_5$ de sorte que la médiane diminue et la moyenne augmente ? Expliquez la démarche.
	\item Peut-on ajouter 1 valeur à la série $S_6$ de sorte que la médiane égale le mode ? Que se passerait-il pour la moyenne ? Expliquez la démarche.
\end{enumerate}

\hyperref[sol:niveau17]{Voir solutions page \pageref{sol:niveau17}}.

\newpage

\subsection{Exercice 18 (secondaire : lycée) : comment contribuer efficacement à un projet collaboratif ?}

\label{calc:niveau18}

Considérons deux contributeurs de Wikipédia : Alice et Bob. La semaine 1, Alice modifie 60\% des articles qu'elle consulte alors que Bob modifie 90\% des articles qu'il lit. La semaine 2, Alice ne modifie que 10\% des articles lus et Bob 30\%. 

\begin{enumerate}[label=C\arabic*)]
	\item Qui a le taux de modifications le plus élevé de la semaine 1 ?
	\item Qui a le taux de modifications le plus élevé de la semaine 2 ?
	\item La semaine 1, Alice lit 100 articles et en modifie 60. Pendant ce temps, Bob modifie 9 des 10 articles qu'il consulte. La semaine 2, Alice modifie 1 article sur les 10 lus et Bob 30 sur 100.  
	
	Sur les deux semaines qui a modifié le plus d'articles ?
	\item Ranger les informations dans un tableau avec 3 colonnes : semaine 1, semaine 2, total et deux lignes : Alice, Bob.
	\item On appelle $S_A(1)$ le taux de modification d'Alice la semaine 1 et $S_A(2)$ son taux de modification la semaine 2. On fait de même pour Bob avec les notations $S_B(1)$ et $S_B(2)$. Vérifier que $S_A(1) < S_B(1)$ et que $S_A(2) < S_B(2)$.
	\item On note les taux sur les deux semaines de la façon suivante :
		\begin{align*}
			S_A &= \dfrac{100}{110}S_A(1) + \dfrac{10}{110}S_A(2) \\
			S_B &= \dfrac{10}{110}S_B(1) + \dfrac{100}{110}S_B(2)
		\end{align*}
	 	Justifiez les valeurs des fractions utilisées.
\end{enumerate}

\hyperref[sol:niveau18]{Voir solutions page \pageref{sol:niveau18}}.

\newpage


\subsection{Exercice 19 (secondaire : lycée) : une interprétation géométrique du paradoxe de Simpson}

\label{geom:niveau19}

\begin{enumerate}[label=C\arabic*)]
	\item Placer les points O(0 ; 0) et A(1 ; 0) et tracer en \textcolor{blue}{bleu} le vecteur \[\textcolor{blue}{\vec{u}_1 = \overrightarrow{OA}}\]
	\item Placer le point B(3 ; 1) et tracer en \textcolor{red}{rouge} le vecteur \[\textcolor{red}{\vec{v}_1 = \overrightarrow{OB}}\]
	\item Placer les points C(0 ; 3) et D(2 ; 7) et tracer en \textcolor{blue}{bleu} le vecteur \[\textcolor{blue}{\vec{u}_2 = \overrightarrow{CD}}\]
	\item Placer le point E(0 ; 4) et tracer en \textcolor{red}{rouge} le vecteur \[\textcolor{red}{\vec{v}_2 = \overrightarrow{CE}}\]
	\item Vérifier que le pente de \textcolor{blue}{$\vec{u}_1$} est supérieure à celle de \textcolor{red}{$\vec{v}_1$}.
	\item Vérifier que le pente de \textcolor{blue}{$\vec{u}_2$} est supérieure à celle de \textcolor{red}{$\vec{v}_2$}.
	\item Placer les points F(4 ; 2) et G(7 ; 4) et tracer en \textcolor{red}{rouge} le vecteur \[\textcolor{red}{\vec{v} = \vec{v}_1 + \vec{v}_2 = \overrightarrow{FG}}\]
	\item Placer le point H(7 ; 6) et tracer en \textcolor{blue}{bleu} le vecteur \[\textcolor{blue}{\vec{u} = \vec{u}_1 + \vec{u}_2  = \overrightarrow{FH}}\]
	\item Comparer les pentes des vecteurs \textcolor{blue}{$\vec{u}$} et \textcolor{red}{$\vec{v}$}. Que remarquez-vous ?
\end{enumerate}

\hyperref[sol:niveau19]{Voir solutions page \pageref{sol:niveau19}}.


\newpage


\subsection{Exercice 20 (secondaire : lycée) : constructions et comparaisons de moyennes}

\label{geom:niveau20}

\begin{enumerate}[label=C\arabic*)]
	\item Placer les points O(0 ; 0), A(4 ; 0) et B(-4 ; 0) puis tracer le demi-cercle de centre O passant par A et B.
	\item Placer le point C(2 ; 0) puis le point D d'abscisse 2 sur le demi-cercle. Tracer le segment [CD].
	\item On pose $a = BC$ et $b = CA$. Ainsi le demi-cercle a pour diamètre $a + b$. Placer le point E(0 ; 4). Montrer que \[OE = \dfrac{a + b}{2}\]
	\item Montrer que le triangle ADB est rectangle en D.
	\item Exprimer AD en fonction CD et b.
	\item Exprimer BD en fonction de CD et a.
	\item En déduire une expression de CD en fonction de a et b.
	\item On appelle moyenne géométrique de a et b le nombre $\sqrt{ab}$ et moyenne arithmétique le nombre $\dfrac{a + b}{2}$. Utilisez ce qui précède pour démontrer qu'on a : \[\dfrac{a+b}{2} \geq \sqrt{ab}\]
	\item Tracer OD puis la hauteur issue de C qui coupe (OD) en F. 
	\item Exprimer OC en fonction de a et b.
	\item Calculer l'aire du triangle DOC de deux façons différentes. D'une part en utilisant OC comme base et CD comme hauteur, d'autre part en utilisant OD comme base et FC comme hauteur. En déduire une expression de FC en fonction de a et b.
	\item Montrer \[FD = \dfrac{2ab}{a + b}\] c'est ce qu'on appelle la moyenne harmonique de a et b.
	\item Tracer EC et calculer sa longueur. Montrer que \[EC = \sqrt{\dfrac{a^2 + b^2}{2}}\]
	On l'appelle moyenne quadratique de a et b.
	\item Classer les différentes moyennes dans l'ordre croissant en utilisant uniquement la géométrie.
	\item Faire de même en utilisant uniquement les calculs algébriques.
\end{enumerate}

\hyperref[sol:niveau20]{Voir solutions page \pageref{sol:niveau20}}.