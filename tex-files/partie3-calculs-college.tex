\section{Calculs niveau collège}
\addcontentsline{toc}{chapter}{Calculs niveau collège}

Dans cette partie, on commence dans la continuité de la partie précédente avec des calculs purs et durs mais déjà vous devriez pouvoir trouver des astuces de calculs. Cela signifie que le raisonnement commence à jouer un rôle utile. Ensuite, on passera au calcul littéral, puis sa liaison avec la géométrie. Et de la géométrie on en viendra aux probabilités. 

\newpage

\subsection{Exercice 7 (secondaire : collège) : carrés}


\label{calc:niveau7}

\begin{enumerate}[label=C\arabic*)]
    \item \(11^2 = 11 \times 11 =  \_\_\_\)
    \item \(12^2 = 12 \times 12 = \_\_\_\)
    \item \(13^2 = 13 \times 13 = \_\_\_\)
    \item \(14^2 = 14 \times 14 = \_\_\_\)
    \item \(15^2 = 15 \times 15 = \_\_\_\)
    \item \(16^2 = 16 \times 16 = \_\_\_\)
    \item \(17^2 = 17 \times 17 = \_\_\_\)
    \item \(18^2 = 18 \times 18 = \_\_\_\)
    \item \(19^2 = 19 \times 19 = \_\_\_\)
    \item \(20^2 = 20 \times 20 = \_\_\_\)
    \item \(25^2 = 25 \times 25 =  \_\_\_\)
    \item \(35^2 = 35 \times 35 = \_\_\_\)
    \item \(45^2 = 45 \times 45 = \_\_\_\)
    \item \(55^2 = 55 \times 55 = \_\_\_\)
    \item \(65^2 = 65 \times 65 = \_\_\_\)
    \item \(75^2 = 75 \times 75 = \_\_\_\)
    \item \(85^2 = 85 \times 85 = \_\_\_\)
    \item \(95^2 = 95 \times 95 = \_\_\_\)
    \item \(111^2 = 111 \times 111 = \_\_\_\)
    \item \(1111^2 = 1111 \times 1111 = \_\_\_\)
\end{enumerate}




\hyperref[sol:niveau7]{Voir solutions des calculs de carrés page \pageref{sol:niveau7}}.

\newpage

\subsection{Exercice 8 (secondaire : collège) : carrés avec des 1 et une calculatrice}


\label{calc:niveau8}

\begin{enumerate}[label=C\arabic*)]
    \item \(1^2 = 1 \times 1 =  \_\_\_\)
    \item \(11^2 = 11 \times 11 = \_\_\_\)
    \item \(111^2 = 111 \times 111 = \_\_\_\)
    \item \(1111^2 = 1111 \times 1111 = \_\_\_\)
    \item \(11111^2 = 11111 \times 11111 = \_\_\_\)
    \item \(111111^2 = 111111 \times 111111 = \_\_\_\)
    \item \(1111111^2 = 1111111 \times 1111111 = \_\_\_\)
    \item \(11111111^2 = 11111111 \times 11111111 = \_\_\_\)
    \item \(111111111^2 = 111111111 \times 111111111 = \_\_\_\)
    \item \(1111111111^2 = 1111111111 \times 1111111111 = \_\_\_\)
\end{enumerate}

\hyperref[sol:niveau8]{Voir solutions des calculs de carrés page \pageref{sol:niveau8}}.


\newpage 

\subsection{Exercice 9 (secondaire : collège) : Un programme de calcul particulier}

\label{calc:niveau9}

Considérons le programme de calcul suivant :

\begin{enumerate}[label=P\arabic*)]
	\item Choisir un nombre entier strictement supérieur à 1 (\(m > 1\) par exemple 2, 3, 4\dots ) 
	\item Choisir un autre nombre entier strictement positif \(n\) strictement inférieur à \(m\quad n < m\)
	\item Calculer le nombre \(a\) comme la différence du carré de \(m\) avec le carré de \(n\), concrètement \(a = m^2 - n^2\)
	\item Calculer le nombre \(b\) comme le double du produit de \(m\) et de \(n\), concrètement \(b = 2mn\)
	\item Calculer le nombre \(c\) comme la somme du carré de \(m\) avec le carré de \(n\), concrètement \(c = m^2 + n^2\)
	\item Calculer le carré de \(a\)
	\item Calculer le carré de \(b\)
	\item Calculer le carré de \(c\)
	\item Calculer la somme du carré de \(a\) et celui de \(b\)
	\item Comparer cette somme avec le carré de \(c\)
\end{enumerate}

\newpage

Appliquons le programme de calcul ci-dessus avec les nombres \(m = 2\) et \(n = 1\) : 

\begin{enumerate}[label=C\arabic*)]
    \item \(m^2 = 2 \times 2 =  \_\_\_\)
    \item \(n^2 = 1 \times 1 = \_\_\_\)
    \item \(a = m^2 - n^2 = \_\_\_\)
    \item \(b = 2 \times m \times n = \_\_\_\)
    \item \(c = m^2 + n^2 = \_\_\_\)
    \item \(a^2 =  \_\_\_\)
    \item \(b^2 =  \_\_\_\)
    \item \(c^2 =  \_\_\_\)
    \item Vérifiez que \( a^2 + b^2 = c^2 \)
    \item Est-ce valable pour n'importe quelles valeurs de \(m\) et \(n\) ?
\end{enumerate}


\hyperref[sol:niveau9]{Voir solutions des calculs du programme page \pageref{sol:niveau9}}.

\newpage 

\subsection{Exercice 10 (secondaire : collège) : Un programme de construction géométrique, Pythagore}

\label{geom:niveau10}

\begin{enumerate}[label=G\arabic*)]
	\item Dans un repère orthonormé placer le point \(A\) de coordonnées \((-1 ; -2)\) c'est-à-dire d'abscisse \(x_A = -1\) et d'ordonnée \(y_A = -2\)
	\item Placer le point \(B\) de coordonnées \((2 ; -2)\) c'est-à-dire d'abscisse \(x_B = 2\) et d'ordonnée \(y_B = -2\)
	\item Placer le point \(C\) de coordonnées \((2 ; 2)\) c'est-à-dire d'abscisse \(x_C = 2\) et d'ordonnée \(y_C = 2\)
	\item En utilisant le théorème de Pythagore vérifier que le carré ABC est rectangle en B.
\end{enumerate}

\hyperref[sol:niveau10]{Voir solution du programme de construction géométrique page \pageref{sol:niveau10}}.

\newpage


\subsection{Exercice 11 (secondaire : collège) : Un cible circulaire, probabilités}

\label{proba:niveau11}

Considérons la cible définie par les cercles concentriques sur le schéma ci-dessous : 
\definecolor{yellow}{rgb}{1,0.8,0.2}
\definecolor{orange}{rgb}{1,0.5,0}
\definecolor{navy}{rgb}{0,0,1}
\definecolor{lime}{rgb}{0,1,0}
\definecolor{red}{rgb}{1,0,0}
\definecolor{black}{rgb}{0,0}

\begin{figure}[H]
\centering
\caption{Cible de cercles concentriques}
\vspace{.25cm}
\begin{tikzpicture}[line cap=round,line join=round,>=triangle 45,x=1cm,y=1cm]
\begin{axis}[
x=1cm,y=1cm,
axis lines=middle,
ymajorgrids=true,
xmajorgrids=true,
xmin=-5.5,
xmax=5.5,
ymin=-5.5,
ymax=5.5,
xtick={-5, -4,...,5},
ytick={-5,-4,...,5},]
\clip(-5.5,-5.5) rectangle (6,5.5);
\draw [line width=2pt,color=red] (0,0) circle (1cm);
\draw [line width=2pt,color=lime] (0,0) circle (2cm);
\draw [line width=2pt,color=navy] (0,0) circle (3cm);
\draw [line width=2pt,color=orange] (0,0) circle (4cm);
\draw [line width=2pt,color=yellow] (0,0) circle (5cm);
\begin{scriptsize}
\draw [fill=black] (0,0) circle (2.5pt);
\draw[color=black] (0.14280352603970045,0.37709534368070763) node {$O$};
\draw [fill=red] (1,0) circle (2.5pt);
\draw[color=red] (1.25,0.38) node {$A$};
\draw[color=red] (1.05,1.05) node {$(\mathcal{C}_A)$};
\draw [fill=lime] (2,0) circle (2.5pt);
\draw[color=lime] (2.25,0.38) node {$B$};
\draw[color=lime] (1.75,1.75) node {$(\mathcal{C}_B)$};
\draw [fill=navy] (3,0) circle (2.5pt);
\draw[color=navy] (3.25,0.378) node {$C$};
\draw[color=navy] (2.5,2.5) node {$(\mathcal{C}_C)$};
\draw [fill=orange] (4,0) circle (2.5pt);
\draw[color=orange] (4.25,0.378) node {$D$};
\draw[color=orange] (3.15,3.15) node {$(\mathcal{C}_D)$};
\draw [fill=yellow] (5,0) circle (2.5pt);
\draw[color=yellow] (5.25,0.378) node {$E$};
\draw[color=yellow] (3.9,3.9) node {$(\mathcal{C}_E)$};
\end{scriptsize}
\end{axis}
\end{tikzpicture}
\label{fig:proba-target}
\end{figure}

\newpage

\definecolor{yellow}{rgb}{1,0.8,0.2}
\definecolor{orange}{rgb}{1,0.5,0}
\definecolor{navy}{rgb}{0,0,1}
\definecolor{lime}{rgb}{0,1,0}
\definecolor{red}{rgb}{1,0,0}
\definecolor{black}{rgb}{0,0}

On considère que les joueurs atteignent toujours la cible c'est-à-dire le cercle $\textcolor{yellow}{(\mathcal{C}_E)}$.
\begin{enumerate}[label=G\arabic*)]
\item Quelle est la probabilité que le joueur atteigne l'intérieur du cercle $\textcolor{red}{(\mathcal{C}_A)}$ ?
\item Quelle est la probabilité que le joueur atteigne la couronne entre les cercles $\textcolor{red}{(\mathcal{C}_A)}$ et $\textcolor{lime}{(\mathcal{C}_B)}$ ?
\item Quelle est la probabilité que le joueur atteigne la couronne entre les cercles $\textcolor{lime}{(\mathcal{C}_B)}$ et $\textcolor{navy}{(\mathcal{C}_C)}$ ?
\item Quelle est la probabilité que le joueur atteigne la couronne entre les cercles $\textcolor{navy}{(\mathcal{C}_C)}$ et $\textcolor{orange}{(\mathcal{C}_D)}$ ?
\item Quelle est la probabilité que le joueur atteigne la couronne entre les cercles $\textcolor{orange}{(\mathcal{C}_D)}$ et $\textcolor{yellow}{(\mathcal{C}_E)}$ ?
\end{enumerate}


\hyperref[sol:niveau11]{Voir solutions page \pageref{sol:niveau11}}.

\newpage


\subsection{Exercice 12 (secondaire : collège) : carrés de Fibonacci}

\label{geom:niveau12}

\begin{enumerate}[label=G\arabic*)]
	\item Dans un repère orthonormé construire le carré passant par les points O(0; 0) , A(1 ; 0) , B(1 ; 1) , C(0 ; 1). Quelle est la longueur du côté de carré ?
	\item Placer les points D(2 ; 0) et E(2 ; 1) et tracer le carré ADEB. Quelle est la longueur du côté de carré ?
	\item Construire le carré passant par F(2 ; 3) , G(0 ; 3) , C(0 ; 1), E(2 ; 1). Quelle est la longueur du côté de carré ?
	\item Placer les points H$(-3 ; 3)$, I$(-3 ; 0)$ et tracer le carré GHIO. Quelle est la longueur du côté de carré ?
	\item Construire le carré passant par J$(-3 ; -5)$, K$(2 ; -5)$, D$(2 ; 0)$, I$(-3 ; 0)$. Quelle est la longueur du côté de carré ?
	\item Placer les points L$(10 ; -5)$, M$(10 ; 3)$ et tracer le carré KLMF. Quelle est la longueur du côté de carré ?
	\item Construire le carré passant par N$(10 ; 16)$, P$(-3 ; 16)$ et tracer le carré MNPH. Quelle est la longueur du côté de carré ?
	\item Placer les points Q$(-24 ; 16)$, R$(-24 ; -5)$ et tracer le carré PQRJ. Quelle est la longueur du côté de carré ?
\end{enumerate}

\hyperref[sol:niveau12]{Voir solutions page \pageref{sol:niveau12}}.

\newpage


\subsection{Exercice 13 (secondaire : collège) : aire des carrés de Fibonacci}

\label{geom:niveau13}

Dans cet exercice on reprend la figure des carrés de Fibonacci voir page \pageref{sol:niveau12}.

\begin{enumerate}[label=G\arabic*)]
	\item Quelle est l'aire du carré \textcolor{red}{OABC} ?
	\item Quelle est l'aire du carré \textcolor{red}{ADEB} ?
	\item Quelle est l'aire du carré \textcolor{orange}{FGCE} ?
	\item Quelle est l'aire du carré \textcolor{olive}{GHIO} ?
	\item Quelle est l'aire du carré \textcolor{navy}{IJKD} ?
	\item Quelle est l'aire du carré \textcolor{pink}{KLMF} ?
	\item Quelle est l'aire du carré \textcolor{lime}{MNPH} ?
	\item Quelle est l'aire du carré \textcolor{purple}{PQRJ} ?
\end{enumerate}


\hyperref[sol:niveau13]{Voir solutions page \pageref{sol:niveau13}}.


\newpage 


\subsection{Exercice 14 (secondaire : collège) : une cible avec des carrés de Fibonacci}

\label{proba:niveau14}

Dans cet exercice on continue avec la figure des carrés de Fibonacci voir page \pageref{sol:niveau12}.
On la considère comme une cible particulière et on admet que le joueur atteint forcément le grand rectangle.

\begin{enumerate}[label=G\arabic*)]
	\item Quelle est la probabilité que le joueur atteigne le carré \textcolor{red}{OABC} ? 
	\item Quelle est la probabilité que le joueur atteigne le carré \textcolor{orange}{FGCE} ?
	\item Quelle est la probabilité que le joueur atteigne le carré \textcolor{olive}{GHIO} ?
	\item Quelle est la probabilité que le joueur atteigne le carré \textcolor{navy}{IJKD} ?
	\item Quelle est la probabilité que le joueur atteigne le carré \textcolor{pink}{KLMF} ?
	\item Quelle est la probabilité que le joueur atteigne le carré \textcolor{lime}{MNPH} ?
	\item Quelle est la probabilité que le joueur atteigne le carré \textcolor{purple}{PQRJ} ?    
\end{enumerate}

\hyperref[sol:niveau14]{Voir solutions page \pageref{sol:niveau14}}.

\newpage


